\documentclass{article}
\usepackage{amsmath}
\usepackage{amsfonts}
\usepackage{amssymb}
\usepackage{geometry}
\geometry{a4paper, margin=1in}
\title{A Proposed Fundamental Equation for Chiral Harmonic Particle Theory}
\author{Gemini}
\date{August 2025}

\begin{document}

\maketitle

\section{Objective}
Our goal is to formulate a Lorentz-invariant dynamical equation for a single fundamental field that can account for the core tenets of the Chiral Harmonic Particle Theory (CHPT). These tenets include:
\begin{enumerate}
    \item A stable, low-density, expansive vacuum state ($\rho_0$).
    \item A spectrum of stable, localized, high-density field configurations representing particles.
    \item The incorporation of chirality.
\end{enumerate}
We select a complex scalar field, $\Phi(x^\mu)$, as our fundamental variable. The field's magnitude squared, $\rho = |\Phi|^2 = \Phi^*\Phi$, represents the local field density. The complex nature of $\Phi$ allows for an internal U(1) symmetry, which is the simplest representation of chirality or internal phase.

\section{The Lagrangian Formulation}
We begin by positing a Lagrangian density, $\mathcal{L}$, which is a function of the field $\Phi$ and its spacetime derivatives $\partial_\mu \Phi$. The Lagrangian must be a Lorentz scalar. The general form is:
\begin{equation}
    \mathcal{L} = (\partial_\mu \Phi^*)(\partial^\mu \Phi) - V(|\Phi|)
\end{equation}
Here, the first term is the kinetic energy term, ensuring relativistic wave propagation. The second term, $V(|\Phi|)$, is a potential energy function that depends only on the field's magnitude and will govern the static solutions and particle spectrum.

The dynamics are derived from the Euler-Lagrange equation:
\begin{equation}
    \partial_\mu \left( \frac{\partial \mathcal{L}}{\partial(\partial_\mu \Phi^*)} \right) = \frac{\partial \mathcal{L}}{\partial \Phi^*}
\end{equation}
Applying this to our Lagrangian gives:
\begin{align}
    \frac{\partial \mathcal{L}}{\partial(\partial_\mu \Phi^*)} &= \partial^\mu \Phi \\
    \frac{\partial \mathcal{L}}{\partial \Phi^*} &= -\frac{\partial V}{\partial \Phi^*} = -\frac{dV}{d|\Phi|} \frac{\partial |\Phi|}{\partial \Phi^*} = -\frac{dV}{d|\Phi|} \frac{\Phi}{2|\Phi|}
\end{align}
Substituting these back into the Euler-Lagrange equation yields the equation of motion:
\begin{equation}
    \partial_\mu \partial^\mu \Phi + \frac{1}{2|\Phi|} \frac{dV}{d|\Phi|} \Phi = 0
\end{equation}
Recognizing the d'Alembertian operator $\Box \equiv \partial_\mu \partial^\mu$, and letting $V' \equiv dV/d|\Phi|$, we arrive at the fundamental equation:
\begin{equation}
    \Box \Phi + \frac{V'(|\Phi|)}{2|\Phi|} \Phi = 0
    \label{eq:eom}
\end{equation}

\section{Engineering the CHPT Potential}
The physics of the theory is encoded entirely within the potential $V(|\Phi|)$. We must engineer its shape to meet the requirements of CHPT. A polynomial in $\rho = |\Phi|^2$ is a versatile choice. To achieve at least one vacuum state and one particle state, we need a potential with at least two distinct, stable minima. A 6th-order polynomial in $|\Phi|$ (or a 3rd-order polynomial in $\rho$) is a suitable candidate.

Let us define the potential in terms of the field density $\rho$:
\begin{equation}
    V(\rho) = \Lambda + c\rho - \frac{b}{2}\rho^2 + \frac{a}{3}\rho^3
\end{equation}
where $a, b, c$ are positive constants and $\Lambda$ is a cosmological constant term.

The extrema of the potential (the possible static states) are found by setting $\frac{dV}{d\rho} = 0$:
\begin{equation}
    \frac{dV}{d\rho} = c - b\rho + a\rho^2 = 0
\end{equation}
The solutions for the density $\rho$ are:
\begin{equation}
    \rho_{\pm} = \frac{b \pm \sqrt{b^2 - 4ac}}{2a}
\end{equation}
By choosing the constants such that $b^2 > 4ac$, we obtain two real, positive density values for our stable states.

\subsection{Interpretation of the States}
\begin{enumerate}
    \item \textbf{Vacuum State ($\rho_0$):} The lower-density solution, $\rho_0 = \rho_-$, corresponds to the CHPT vacuum. This is the state of the field in "empty" space. To account for the "expansive tendency" (dark energy), the value of the potential at this minimum must be positive, i.e., $V(\rho_0) > 0$. We can achieve this by tuning the cosmological constant term $\Lambda$. The effective pressure of this state is $p_0 = \mathcal{L}_0 = \rho_0(dV/d\rho)|_{\rho_0} - V(\rho_0) = -V(\rho_0)$, which is negative, driving expansion.
    \item \textbf{Particle State ($\rho_p$):} The higher-density solution, $\rho_p = \rho_+$, corresponds to the core density of a fundamental particle. A stable particle is a localized soliton solution where the field is concentrated at density $\rho_p$ in a finite region and decays to the vacuum value $\rho_0$ at infinity.
\end{enumerate}

The mass of such a particle (soliton) would be given by the total energy of this field configuration:
\begin{equation}
    M_p = \int \left[ |\nabla\Phi|^2 + V(|\Phi|) \right] d^3x
\end{equation}
By adding more terms to the polynomial potential (e.g., up to $\rho^4, \rho^5$), one could create additional minima, corresponding to a spectrum of different fundamental particles with different core densities and thus different masses, as envisioned by CHPT.

\section{Conclusion: The Proposed Equation}
The proposed fundamental equation is the Nonlinear Klein-Gordon equation,
\begin{equation}
    \boxed{ \Box \Phi + \left( c - b|\Phi|^2 + a|\Phi|^4 \right) \Phi = 0 }
\end{equation}
where we have substituted the derivative of our proposed potential $V(|\Phi|)$ into the general equation of motion (\ref{eq:eom}). This equation provides a concrete, relativistic, and mathematically sound starting point for quantifying the Chiral Harmonic Particle Theory. It is capable of modeling a stable, expansive vacuum and stable, localized particle states from a single, unified complex scalar field, thereby addressing the principal weakness of the conceptual framework. Further research would involve finding the soliton solutions to this equation and investigating if their topological properties can be mapped to the quantum numbers of the Standard Model.

\end{document}